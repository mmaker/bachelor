\documentclass[12pt,a4paper,twoside]{thesis}

%% PACKAGES
\usepackage[utf8]{inputenc}
\usepackage[T1]{fontenc}
\usepackage{algorithm}
\usepackage[noend]{algpseudocode}
\usepackage{amsmath}
\usepackage{amsthm}
\usepackage{amsfonts}
\usepackage{amssymb}
\usepackage{amsthm}
\usepackage{bytefield}
\usepackage{cancel}
\usepackage[dvips]{color}
\usepackage{enumerate}
\usepackage{epigraph}
\usepackage{fancyhdr}
\usepackage{graphicx}
\usepackage{hyperref}
\usepackage{indentfirst}
\usepackage{mathtools}
\usepackage{minted}
\usepackage{makeidx,shortvrb,latexsym}
\usepackage{supertabular}
\usepackage{tikz}

%% setting epigraphs
\renewcommand{\epigraphsize}{\small}
\setlength{\epigraphwidth}{0.8\textwidth}
\let\origepigraph\epigraph
\renewcommand\epigraph[2]{\origepigraph{\textit{#1}}{\textsc{#2}}}

%% bytefield shit
\newcommand{\colorbitbox}[3]{%
\rlap{\bitbox{#2}{\color{#1}\rule{\width}{\height}}}%
\bitbox{#2}{#3}}


%% COMMANDS
\DeclarePairedDelimiter{\floor}{\lfloor}{\rfloor}
\DeclarePairedDelimiter{\ceil}{\lceil}{\rceil}
\DeclarePairedDelimiter{\angular}{\langle}{\rangle}

\newcommand{\naturalN}{\mathbb{N}}
\newcommand{\naturalPrime}{\mathbb{P}}
\newcommand{\integerZ}{\mathbb{Z}}
\newcommand{\bigO}[1]{\ensuremath{\operatorname{O}\left(#1\right)}}
\newcommand{\openssl}{\textsc{OpenSSL}\ }
%% here adopting Wikipedia's notation <https://en.wikipedia.org/wiki/Isqrt>
\newcommand{\dsqrt}[1]{\ensuremath{isqrt(#1)}}
\newcommand{\idiv}{\ensuremath{//}}
\newcommand{\strong}[1]{\textbf{#1}}
\newcommand{\eulerphi}[1]{\varphi(#1)}
\newcommand{\abs}[1]{\left|#1\right|}
\newcommand{\rfrac}[2]{{}^{#1}\!/_{#2}}
\newcommand{\getsRandom}{\xleftarrow{r}}

\makeindex
\let\origdoublepage\cleardoublepage
\newcommand{\clearemptydoublepage}{%
\clearpage
{\pagestyle{empty}\origdoublepage}%
}
\let\cleardoublepage\clearemptydoublepage
% Note that colour boxes around links are NOT printed.
% The text itself can be coloured, replacing the bounding box, but
% when printing this may appear illegible;
% link colors can be set to black for printing purposes, like so:
%\hypersetup{colorlinks,%
%	citecolor=black,%
%	filecolor=black,%
%	linkcolor=black,%
%	urlcolor=black}
% Also note that this will conflict with the color package called
% earlier in this document if that is not set to the right option (e.g.
% dvips or pdflatex).
\begin{document}
%\phd  %se e' la tesi di dottorato, altrimenti non mettere nulla

\university{Universit\`a degli Studi di Trento}
\faculty{Facolt\`a di Scienze Matematiche Fisiche e Naturali}
\dept{Dipartimento di Scienze Informatiche}
\Logo{logo_unitn.jpg}
%% titolo del dottorato
% \phdtitle{}
%% titolo della tesi
\title{Question Authority}
\subtitle{An Inquiry into The Secure Layer}

\author{Michele Orr\`u}
\supervisor{Prof. Massimiliano Sala}
%% Uncomment the following two lines if a co-relator is present.
%\twosupervisors
%\firstreader{Prof. Ettore Miron}

% capo della scuola di dottorato o controrelatore
\secondreader{Prof. Nara Stabbocchi}
\accademico{Anno accademico $2013/2014$}

\frontespizio     % questo e` il frontespizio esterno, cioe' senza firme
\cleardoublepage
\signaturepage    % questo e` il frontesizio interno con le firme

%% DEDICA
%non e' ovviamente obbligatoria


\cleardoublepage \setcounter{page}{1} \pagenumbering{roman}
\pagestyle{plain} \tableofcontents
%\listoffigures
%\addcontentsline{toc}{chapter}{Elenco delle figure}
%\renewcommand{\listalgorithmname}{Elenco degli algoritmi}
%\listofalgorithms
%\addcontentsline{toc}{chapter}{Elenco degli algoritmi}
%\addcontentsline{toc}{chapter}{Introduction}
%% INTRODUCTION
%\include{ack}
%\addcontentsline{toc}{chapter}{Acknowledgment}
%\cleardoublepage
%\include{Introduction}
%\addcontentsline{toc}{chapter}{Introduction}
\cleardoublepage

%% THESIS BODY
\pagestyle{fancy} \pagenumbering{arabic} \mainmatter
\vspace*{3in}
\epigraph{
    Many persons who are not conversant with mathematical studies imagine that
    because the business of [Babbage's Analytical Engine] is to give results in
    numerical notation, the nature of the processes must consequently be
    arithmetical and numerical, rather than algebraical and analytical. This is an
    error. The engine can arrange and combine its numerical quantities exactly as if
    they were letters or any other general symbols; and in fact it might bring out
    its results in algebraical notation, were provisions made accordingly.}
         {Augusta Ada, Countess of Lovelace}


%% there is no copyright, but the right to copy.
\begin{figure}[b]
  \centering
  \includegraphics[width=80pt]{kopimi.png}
\end{figure}


\chapter{Preface}

Even if the basic RSA keypair generation algorithm is fairly
straightforward, it turns out that any software willing to provide such a
feature does have to test the pair candidate against a substantious number of
tests before claiming its security.

The purpose of this project is to examine the TLS protocol, study in deep the
\openssl library, and survey some of the attacks to which a bad key generation
is exposed. On the footprint of ~\cite{20years}, we which already analyzed most
of these, we are going to describe the mathematical basis of each attack, and
then proceed further reasoning about a clever, possibly optimal, solution in
procedural programming; finally, we are trying to think about a distributed
version of it.

Besides the pseudocode already available in this document, the project led to the
development of a real, open, C implementation consultable at
\small{\url{https://github.com/mmaker/bachelor}}.

%%% Local Variables:
%%% mode: latex
%%% TeX-master: "question_authority"
%%% End:


\part{Prequisites}
\chapter{The Secure Layer \label{chap:ssl}}

Transport Layer Security, formerly known as SSL (Secure Socket Layer), aims
to bring some security features over a communication channel, specifically
providing \strong{integrity} and \strong{confidentiality} of the message, \strong{authenticity} of the server and
optionally the client.
%% fuck osi layers: there is no code explicitly structuring the internet in 7
%% layers.
The most allocate TLS in the 6 or 7th OSI Layer, ``Application'', and is nowdays widely adopted
all over the world, being the de-facto standard for end-to-end  encryption.

\paragraph{Certifications Authority} are at the root of the security of the
protocol. See section ~\ref{sec:ssl:x509}

\paragraph{The protocol} is actually composed of many sub-protocols:

\begin{itemize}
\item handshake protocol
\item record protocol
\item alert protocol
\item changespec protocol ?
\end{itemize}
We will proceed by describing in deep only the first two of these, due to their
relevant role inside the conection and furthermore, because they are the only
two we actually made use of during our investigations.


\section{The \texttt{handshake} protocol}
Different options:
\begin{itemize}
\item no session
\item session
\item client authenticaton
\end{itemize}


\section{The \texttt{record} protocol}

Until 2005, failure to authenticate, decrypt will result in I/O error and a
close of the connection

\section{What's inside a certificate \label{sec:ssl:x509}}

\section{Remarks among SSL/TLS versions}

cos'e
differenze tra le varie versioni
la certification autority
%%% Local Variables:
%%% mode: latex
%%% TeX-master: "question_authority.tex"
%%% End:

\chapter{Mathematical prequisites \label{chap:preq}}

\section{Euclid's Greatest Common Divisor}

Being the gratest common divisor a foundamental algebraic operation in the ssl
protocol, \openssl implemented it with the following signature:

\begin{minted}[fontsize=\small]{c}
  int BN_gcd(BIGNUM *r, BIGNUM *a, BIGNUM *b, BN_CTX *ctx);
\end{minted}

The computation proceeds under the well-known Euclidean algorithm, specifically
the binary variant developed by Josef Stein in 1961 \cite{AOCPv2}. This variant
exploits some interesting properties of $gcd(u, v)$:

\begin{itemize}
  \setlength{\itemsep}{1pt}
  \setlength{\parskip}{0pt}
  \setlength{\parsep}{0pt}
\item if $u,\ v$ are even, then $gcd(u, v) = 2gcd(u/2, v/2)$
\item if $u$ is even and $v$ is odd, then $gcd(u, v) = gcd(u/2, v)$
\item  $gcd(u, v) = gcd(u-v, v)$, as in the standard Euclid's algorithm
\item the sum of two odd numbers is always even
\end{itemize}

% Donald Knuth, TAOCP, "a binary method", p. 388 VOL 2
Both \cite{AOCPv2} and \cite{MITalg} analyze the running time for the algorithm,
even if \cite{clrs}'s demonstration is fairly simpler and proceeds %elegantly
by induction.
Anyway, both show that algorithm ~\ref{alg:gcd} belongs to the class
\bigO{\log b}.

\begin{algorithm}
  \caption{\openssl's GCD \label{alg:gcd}}
  \begin{algorithmic}[1]
    \State $k \gets 0$
    \While{$v \neq 0$}
      \If{$u$ is odd}
        \If{$v$ is odd}
          \State $a \gets (a-b) \ll 1$
        \Else
          \State $b = b \ll 1$
        \EndIf
        \If{$a < b$} $a, b \gets b, a$ \EndIf

      \Else
        \If{$v$ is odd}
          \State $a = a \ll 1$
          \If{$a < b$} $a, b = b, a$ \EndIf
        \Else
          \State $k = k+1$
          \State $a, b = a \ll 1, b \ll 1$
        \EndIf
      \EndIf
    \EndWhile
    \State \Return $a \ll k$

  \end{algorithmic}
\end{algorithm}


\section{RSA Cipher}

XXX.
define RSA, provide the simple keypair generation algorithm.

From now on, except otherwise specified, the variable $N=pq$ will refer to the
public modulis of a generis RSA keypair, with $p, q\ .\ p > q$ being the two primes
factorizing it. Again, $e, d$ will respectively refer to the public exponent and
the private exponent.


\section{Algorithmic Complexity Notation}
The notation used to describe asymptotic complexity follows the $O$-notation,
abused under the conventions and limits of MIT's Introduction to Algorithms.

Let \bigO{g} be the asymptotic upper bound of g:
$$
O(g(n)) = \{ f(n) : \exists n_0, c \in \naturalN \mid 0 \leq f(n) \leq cg(n)
             \ \forall n > n_0 \}
$$

With the writing $f(n) = O(g(n))$ we will actually interpret
$f(n) \in O(g(n))$.

\section{Square Root \label{sec:preq:sqrt}}

Computing the square root has been another foundamental requirement of the
project, though not satisfied by \openssl. Apprently,
% \openssl is a great pile of crap, as phk states
\openssl does not provide
XXX.
define square root in the algebraic notation
discuss method of computation for square root

%%% Local Variables:
%%% mode: latex
%%% TeX-master: "question_authority"
%%% End:


\part{Factorization Methods}
\chapter{Fermat's factorization method \label{chap:fermat}}

Excluding the trial division, Fermat's method is the oldest known systematic
method for factorizing integers. Even if its algorithmic complexity is not
among the most efficient, it holds still a practical interest whenever
the two primes are sufficiently close.
Indeed, \cite{DSS2009} \S B.3.6 explicitly recommends that
$|p-q| \geq \sqrt{N}2^{-100}$
for any key of bitlength $1024,\ 2048,\ 3072$ in order to address this kind of
threat.\\
%% it would be nice here to explain that this magic 2^100 is just about wonting
%% the most significant digits to be different.
The basic idea is to attempt to write $N$ as a difference of squares,
\begin{align}
\label{eq:fermat_problem}
x^2 - N = y^2
\end{align}

So, we start by $x = \ceil{\sqrt{N}}$ and check that $x^2-N$ is a perfect
square. If it isn't, we iteratively increment $x$ and check again, until we
find a pair $\angular{x, y}$ satisfying equation \ref{eq:fermat_problem}.
Once found, we claim that $N = pq = (x+y)(x-y)$; it is indeed true that, if we
decompose $x^2 - y^2$ as difference of squares, then it is immediately clear
that $x+y \mid N \ \land \  x-y \mid N$, and that both are non-trivial
divisors.

\paragraph{Complexity} \cite{riesel} contains a detailed proof for the
complexity of this algorithm, which is
$\bigO{\frac{(1-k)^2}{2k} \sqrt{N}} \;\;,  0 < k < 1$. We summarize it down
below here to better clarify the limits of this algorithm.

\begin{proof}
  Since, once we reach the final step $x_f$ it holds $N = pq = x_f^2 - y_f^2$,
  the number of steps required to reach the result is:
  \begin{align*}
    x_f - \sqrt{N} &= \frac{p + q}{2} - \sqrt{N} \\
                   &= \frac{p + \frac{N}{p}}{2} - \sqrt{N} \\
                   &= \frac{(\sqrt{N} - p)^2}{2p}
  \end{align*}
  If we finally suppose that $p = k\sqrt{N}, \; 0 < k < 1$, then the number of cycles
  becomes
  $\frac{(1-k)^2}{2k} \sqrt{N}$.
\end{proof}

\begin{remark}
  Note that, for the algorithm to be effective, the two primes must be
  ``really close'' to $\sqrt{N}$. As much as the lowest prime gets near to
  $1$, the ratio $\frac{(1-k)^2}{2k}$ becomes larger, until the actual magnitude
  of this factorization method approaches \bigO{N}.
\end{remark}

\section{An Implementation Perspective \label{sec:fermat:implementation}}

At each iteration, the $i-$th state is hold by the pair $\angular{x, x^2}$.\\
The later step, described by $\angular{x+1, (x+1)^2}$ can be computed efficiently
considering the square of a binomial: $\angular{x+1, x^2 + 2x + 1}$.
The upper-bound, instead, is reached when
$ \Delta = p - q  = x + y - x + y = 2y > 2^{-100}\sqrt{N}$.

Algorithm ~\ref{alg:fermat} presents a simple implementation of this
factorization method, taking into account the small optimizations
aforementioned.

\begin{algorithm}[H]
  \caption{Fermat Factorization \label{alg:fermat}}
  \begin{algorithmic}[1]
    \Function{fermat}{\PKArg}
    \State $x \gets \floor{\sqrt{N}}$
    \State $x' \gets x \cdot x$

    \Repeat
    \State $x' \gets x' + 2x + 1$
    \State $x \gets x+1$
    \State $y, rest \gets \dsqrt{x' - N}$
    \Until{ $rest \neq 0 \strong{ and } y < \frac{\sqrt{N}}{2^{101}}$ }
    \Comment i.e., \ref{eq:fermat_problem} holds?

    \If{ $rest = 0$ }
    \State $p \gets x+y$
    \State $q \gets x-y$
    \State \Return $p, q$
    \Else
    \State \Return \textbf{nil}
    \EndIf
    \EndFunction
    \end{algorithmic}
\end{algorithm}

\paragraph{How to chose the upper limit?}  Our choice of keeping straight with
the limits of the standard is a mere choice of commodity: we are interested in
finding public keys  not respecting the standard.
Though, it is worth noting that what this limit \emph{states} is that at least
one of the most significant $100$ bits should be different between the two
primes:

\begin{bytefield}[
  endianness=big,
  bitwidth=1.35em,
  % bitformatting=\fakerange,
  ]{16}
  \\
  % \bitheader{}
  \\[1px]
  \begin{rightwordgroup}{\small{$2^{\frac{\log N}{2}-100}$}}
    \bitbox{1}{0} & \bitbox{1}{0} & \bitbox{1}{0} & \bitbox{1}{0} &
    \bitbox{1}{0} & \bitbox{1}{0} & \bitbox{1}{0} & \bitbox{1}{0} &
    \bitbox{3}{\tiny $\cdots$} &
    \bitbox{1}{0} & \bitbox{1}{0} & \bitbox{1}{0} & \bitbox{1}{0} &
    \bitbox{1}{0} & \bitbox{1}{1} & \bitbox{1}{0} & \bitbox{1}{0} &
    \bitbox{3}{\tiny $\cdots$}    & \bitbox{1}{0} & \bitbox{1}{0} &
  \end{rightwordgroup}
  \\[1ex]
  \wordbox[]{1}{} &&
  \\[1ex]
  \begin{rightwordgroup}{$p$}
    \bitbox{1}{0} & \bitbox{1}{1} & \bitbox{1}{0} & \bitbox{1}{0} &
    \bitbox{1}{0} & \bitbox{1}{0} & \bitbox{1}{1} & \bitbox{1}{1} &
    \bitbox{3}{\tiny $\cdots$} &
    \bitbox{1}{0} & \bitbox{1}{1} & \bitbox{1}{0} & \bitbox{1}{0} &
    \bitbox{1}{0} &
    \colorbitbox{lightgray}{1}{1} & \colorbitbox{lightgray}{1}{0} &
    \colorbitbox{lightgray}{1}{0} &
    \colorbitbox{lightgray}{4}{\tiny{$\cdots$ LSB $\cdots$}} &
    \colorbitbox{lightgray}{1}{0} &
  \end{rightwordgroup}
  \\[1ex]
  \begin{rightwordgroup}{$q$}
    \bitbox{1}{0} & \bitbox{1}{1} & \bitbox{1}{0} & \bitbox{1}{0} &
    \bitbox{1}{0} & \bitbox{1}{0} & \bitbox{1}{0} & \bitbox{1}{1} &
    \bitbox{3}{\tiny $\cdots$} &
    \bitbox{1}{0} & \bitbox{1}{1} & \bitbox{1}{0} & \bitbox{1}{0} &
    \bitbox{1}{0} &
    \colorbitbox{lightgray}{1}{0} & \colorbitbox{lightgray}{1}{0} &
    \colorbitbox{lightgray}{1}{0} &
    \colorbitbox{lightgray}{4}{\tiny{$\cdots$ LSB $\cdots$}} &
    \colorbitbox{lightgray}{1}{0} &
  \end{rightwordgroup}
\end{bytefield}
\vfill

For example, in the case of a RSA key $1024$, the binary difference between $p$
and $q$ has to be greater than $2^{412}$, which means that, excluding corner-cases
where the remainder is involved, there must be at least one difference in the
top 100 most significant bits for the key to be considered safe.


\section{Thoughts about a parallel solution}

At first glance we might be willing to split the entire interval
$\{ \ceil{\sqrt{N}}, \ldots, N-1 \}$ in equal parts, one per each
node. However, this would not be any more efficient than the trial division
algorithm, and nevertheless during each single iteration, the computational
complexity is dominated by the square root $\dsqrt$ function, which belongs to
the class \bigO{\log^2 N}, as we saw in section ~\ref{sec:preq:sqrt}.
Computing separatedly $x^2$ would add an overhead of the same order of magnitude
\bigO{\log^2 N}, and thus result in a complete waste of resources.

%%% Local Variables:
%%% TeX-master: "question_authority.tex"
%%% End:

\chapter{Wiener's cryptanalysis method \label{chap:wiener}}

Wiener's attack was first published in 1989 as a result of cryptanalysis on the
use of short RSA secret keys ~\cite{wiener}. It exploited the fact that it is
possible to find the private key in \emph{polynomial time} using continued fractions
expansions whenever a good estimate of the fraction $\frac{e}{N}$ is known.
More specifically, given $d < \frac{1}{3} \sqrt[4]{N}$ one can efficiently
recover $d$ only knowing $\angular{N, e}$.

The scandalous implication behind Wiener's attack is that, even if there are
situations where having a small private exponent may be
particularly tempting with respect to performance (for example, a smart card
communication with a computer), they represent a threat to the security of the
cipher.
Fortunately, ~\cite{wiener} \S 9 presents a couple of precautions that make a
RSA key-pair immune to this attack, namely
(i) making $e > \sqrt{N}$ and
(ii) $gcd(p-1, q-1)$ large.

\section{Background on Continued Fractions \label{sec:wiener:cf}}

Let us call \emph{continued fraction} any expression of the form:
%% why \cfrac sucks this much. |-------------------------|
\begin{align*}
a_0 + \frac{1}{a_1
    + \frac{1}{a_2
    + \frac{1}{a_3
    + \frac{1}{a_4 + \ldots}}}}
\end{align*}
Consider now any \emph{finite continued fraction}, conveniently represented with
the sequence
$\angular{a_0, a_1, a_2, a_3,  \ \ldots, a_n}$.
Any number $x \in \mathbb{Q}$ can be represented as a finite continued fraction,
and for each $i < n$ there exists a fraction $\rfrac{h}{k}$ approximating
$x$.
By definition, each new approximation
$$
\begin{bmatrix}
  h_i \\ k_i
\end{bmatrix}
=
\angular{a_0, a_1, \ \ldots, a_i}
$$
is recursively defined as:

\begin{align}
  \label{eq:wiener:cf}
  \begin{cases}
    a_{-1} = 0 \\
    a_i = h_i // k_i \\
  \end{cases}
  \quad
  \begin{cases}
    h_{-2} = 0 \\
    h_{-1} = 1 \\
    h_i = a_i h_{i-1} + h_{i-2}
  \end{cases}
  \quad
  \begin{cases}
    k_{-2} = 1 \\
    k_{-1} = 0  \\
    k_i = a_i k_{i-1} + k_{i-2}
  \end{cases}
\end{align}

Among the prolific properties of such objects, Legendre in 1768 discovered that,
if a continued fraction $f' = \frac{\theta'}{\kappa'}$ is
an underestimate of another one $f = \frac{\theta}{\kappa}$, i.e.
\begin{align}
  \abs{f - f'} = \delta
\end{align}
then for a $\delta$ sufficiently small, $f'$ is \emph{equal} to the $n$-th
continued fraction expansion of $f$, for some $n \geq 0$ (\cite{smeets} \S 2).
Formally,

\begin{theorem*}[Legendre]
  If $f = \frac{\theta}{\kappa}$,  $f' = \frac{\theta'}{\kappa'}$ and
  $\gcd(\theta, \kappa) = 1$, then
  \begin{align}
  \label{eq:wiener:cf_approx}
    \abs{f' - \frac{\theta}{\kappa}} < \delta = \frac{1}{2\kappa^2}
    \quad
    \text{ implies that }
    \quad
    \begin{bmatrix}
      \theta' \\ \kappa'
    \end{bmatrix}
    =
    \begin{bmatrix}
      \theta_n \\ \kappa_n
    \end{bmatrix},
    \quad
    \text{ for some } n \geq 0
  \end{align}
\end{theorem*}

Two centuries later, first Wiener \cite{wiener} and later Dan Boneh
\cite{20years} leveraged this theorem in order to produce an algorithm able to
recover $f$, having $f'$.
The \emph{continued fraction algorithm}  is the following:
\begin{enumerate}[(i)]
  \setlength{\itemsep}{1pt}
  \setlength{\parskip}{0pt}
  \setlength{\parsep}{0pt}
  \item generate the next $a_i$ of the continued fraction expansion of $f'$;
  \item use ~\ref{eq:wiener:cf} to generate the next fraction $\rfrac{h_i}{k_i}$
    equal to $\angular{a_0, a_1, \ldots, a_{i-1}, a_i}$ %% non e` proprio cosi`
  \item check whether $\rfrac{h_i}{k_i}$ is equal to $f$
\end{enumerate}

\section{Continued Fraction Algorithm applied to RSA}

As we saw in ~\ref{sec:preq:rsa}, by construction the two exponents are such that
$ed \equiv 1 \pmod{\varphi(N)}$. This implies that there exists a
$k \in \naturalN \mid ed = k\varphi(N) + 1$. This can be formalized to be
the same problem we formalized in ~\ref{eq:wiener:cf_approx}:
\begin{align*}
  ed = k\varphi(N) + 1 \\
  \abs{\frac{ed - k\eulerphi{N}}{d\eulerphi{N}}} = \frac{1}{d\eulerphi{N}} \\
  \abs{\frac{e}{\eulerphi{N}} - \frac{k}{d}} = \frac{1}{d\eulerphi{N}} \\
\end{align*}

Now we proceed by substituting $\eulerphi{N}$ with $N$, since for large $N$, one
approximates the other. We consider also the difference of the two, limited by
$\abs{\cancel{N} + p + q - 1 - \cancel{N}} < 3\sqrt{N}$.
For the last step, remember that $k < d < \rfrac{1}{3}\sqrt[4]{N}$:

\begin{align*}
  \abs{\frac{e}{N} - \frac{k}{d}} &= \abs{\frac{ed - kN}{Nd}} \\
  &= \abs{\frac{\cancel{ed} -kN - \cancel{k\eulerphi{N}} + k\eulerphi{N}}{Nd}} \\
  &= \abs{\frac{1-k(N-\eulerphi{N})}{Nd}} \\
  &\leq \abs{\frac{3k\sqrt{N}}{Nd}}
  = \frac{3k}{d\sqrt{N}}
  < \frac{3(\rfrac{1}{3}\ \sqrt[4]{N})}{d\sqrt{N}}
  = \frac{1}{d\sqrt[4]{N}} < \frac{1}{2d^2}
\end{align*}

This demonstrates that the hypotesis of ~\ref{eq:wiener:cf_approx} is satisfied,
and allows us to proceed with the continued fraction algorithm to converge to a
solution ~\cite{20years}.

\paragraph{}
We start by generating the $\log N$ continued fraction expansions of
$\frac{e}{N}$, and for each convergent $\frac{k}{d}$,
%% XXX. verify this
which by contruction is already at the lowest terms, we verify if it produces a
factorization of $N$.
First we check that $\eulerphi{N} = \frac{ed-1}{k}$ is
an integer. Then we solve ~\ref{eq:wiener:pq} in $x$ in order to find $p, q$:
\begin{align}
  \label{eq:wiener:pq}
  x^2 - (N - \eulerphi{N} + 1)x + N = 0
\end{align}
The above equation is constructed so that the $x$ coefficient is the sum of the
two primes, while the constant term $N$ is the product of the two. Therefore, if
$\eulerphi{N}$ has been correctly guessed, the two roots will be $p$ and $q$.

\section{An Implementation Perspective}

The algorithm is pretty straightforward by itself: we just need to apply the
definitions provided in ~\ref{eq:wiener:cf} and test each convergent until
$\log N$ iterations have been reached.
%% XXX. questo viene da 20 years, ma non e` spiegato perche`.
A Continued fraction structure may look like this:

\begin{minted}{c}
  typedef struct cf {
    bigfraction_t fs[3];  /* holding h_i/k_i, h_i-1/k_i-1, h_i-2/k_i-2 */
    short i;              /* cycling in range(0, 3) */
    bigfraction_t x;      /* pointer to the i-th fraction in fs */
    BIGNUM* a;            /* current a_i */
    BN_CTX* ctx;
  } cf_t;
\end{minted}
where \texttt{bigfraction\_t} is just a pair of \texttt{BIGNUM} \!s
$\angular{h_i, k_i}$. Whenever we need to produce a new convergent, we increment
$i \pmod{3}$ and apply the definitions given. The fresh convergent must be
tested with very simple algebraic operations. It is worth noting here that
\ref{eq:wiener:pq} can be solved using the reduced discriminant formula, as
$p, q$ are odd primes:
\begin{align*}
\Delta = \left( \frac{N-\eulerphi{N} + 1}{2} \right)^2 - N \\
x_{\angular{p , q}} = - \frac{N - \eulerphi{N} + 1}{2} \pm \sqrt{\Delta}
\end{align*}
Assuming the existence of the procedures \texttt{cf\_init}, initializing a
continued fraction structure, and \texttt{cf\_next} producing the next
convergent, we provide an algorithm for attacking the RSA cipher via Wiener:

\begin{algorithm}[H]
  \caption{Wiener's Attack}
  \label{alg:wiener}
  \begin{algorithmic}[1]
    \Function{wiener}{\PKArg}
    \State $f \gets  \texttt{cf\_init}(e, N)$
    \For{$\ceil{\log N} \strong{ times }$}
      \State $k, d \gets \texttt{cf\_next}(f)$
      \If{$k \nmid ed-1$} \strong{continue} \EndIf
      \State $\eulerphi{N} \gets (ed - 1)\ //\ k$
      \If{$\eulerphi{N}$ is odd} \strong{continue} \EndIf
%% XXX. it could be that calling 'b' b/2 and 'delta' sqrt(delta/4) is
%% misleading.
      \State $b \gets (N - \eulerphi{N} + 1) \gg 1$
      \State $\Delta, r \gets \dsqrt{b^2 - N}$
      \If{$r \neq 0$} \strong{continue} \EndIf
      \State $p \gets b + \Delta$
      \State $q \gets b - \Delta$
      \State \strong{break}
    \EndFor
    \State \Return p, q
    \EndFunction
  \end{algorithmic}
\end{algorithm}

\paragraph{Parallelism}
Parallel implementation of this specific version of Wiener's Attack is
difficult, because the inner loop is inherently serial. At best, parallelism
could be employed to split the task into a \emph{constructor} process, building
the $f_n$ convergents, and many \emph{consumers} receiving each convergent to be
processed seperatedly.
The first one arriving to a solution, broadcasts a stop message to the others.

%%% Local Variables:
%%% mode: latex
%%% TeX-master: "question_authority"
%%% End:

\include{pollard-1}
\chapter{Pollard's $p+1$ factorization method}

pollard!
%%% Local Variables:
%%% mode: latex
%%% TeX-master: "question_authority"
%%% End:

\chapter{Pollard's $\rho$ factorization method \label{chap:pollardrho}}

Pollard's $\rho$ factorization method is based on the statistical idea behind
the birthday paradox. It consists into indentifying a periodically recurrent
sequence of integers in the ring of remainders with respect to the public
modulus $N$, and claim that the period $\psi$ is one of the two primes
factorizing $N$.

\paragraph{Origins of the name} The $\rho$ name is devoted to the graphical
representation of the algorithm: as we can see in figure ~\ref{fig:pollardrho},
if we graphically represent the lookup over a graphic

\begin{center}
  \begin{tikzpicture}[scale=0.7, thick]
    \tikzstyle{every node}=[draw,circle,fill=white,minimum size=4pt,
                            inner sep=0pt]
    \node (1) at (1.4, 0.2) [label=left:$x_1$] {};
    \node (2) at (2.5, 3)   [label=left:$x_{i-2}$] {};
    \node (3) at (3.25, 5)  [label=left:$x_{i-1}$] {};
    \node (4) at (4, 7)     [label=left:$ x_i \equiv x_j $] {};
    \node (5) at (6, 9)     [label=above:$x_{i+1}$] {};
    \node (6) at (8, 7)     [label=right:$x_{i+2}$] {};
    \node (7) at (6, 5)     [label=below:$x_{j-1}$] {};

    \path (1) edge [dashed] (2);
    \path (2) edge (3);
    \path (3) edge (4);
    \path (4) edge [bend left] (5);
    \path (5) edge [bend left] (6);
    \path (6) edge [bend left, dashed] (7);
    \path (7) edge [bend left] (4);

    %%\draw [decorate,decoration={brace, raise=1.5cm}] (1) -- (3)
    %%node[draw=no] at (-1.5, 4) {tail};
    \draw [decorate,decoration={brace, raise=3cm}] (5) -- (7)
    node[draw=none] at (13, 7) {\footnotesize {periodic sequence}};

\end{tikzpicture}
\end{center}


\paragraph{A more rigourous description}
\begin{proof}
\end{proof}

\section{A Computer program for Pollard's $\rho$ method}

Using the same trick we saw in section ~\ref{sec:pollard-1:implementing},  we
chose to apply occasionally Euclid's algorithm by computing the accumulated
product; algorithm ~\ref{alg:pollardrho} outlines what we have so far discussed,
considering also the pascal transcript present in ~\cite{riesel} \S 5.

\begin{algorithm}
  \caption{Pollard's $\rho$ factorization \label{alg:pollardrho}}
  \begin{algorithmic}[1]
    \State $a \getsRandom \naturalN \setminus \{0, 2\}$
    \State $x \getsRandom \naturalN$
    \State $y \gets x$
  \end{algorithmic}
\end{algorithm}
%%% Local Variables:
%%% mode: latex
%%% TeX-master: "question_authority"
%%% End:

\chapter{Dixon's factorization method\label{chap:dixon}}

~\cite{dixon} describes a class of ``probabilistic algorithms'' for finding a
factor of any composite number, at a sub-exponential cost. They basically
consists into taking random integers $r$ in $\{1, \ldots, N\}$ and look for those
where $r^2 \mod{N}$ is \emph{smooth}. If enough are found, then those integers
can somehow be assembled, and so a fatorization of N attemped.

%% that's not really academic to be stated officially, but I would have never
%% understood this section without Firas (thanks).
%% <http://blog.fkraiem.org/2013/12/08/factoring-integers-dixons-algorithm/>
%% I kept the voila` phrase, that was so lovely.
\section{A little bit of History \label{sec:dixon:history}}
During the latest century there has been a huge effort to approach the problem
formulated by Fermat ~\ref{eq:fermat_problem} from different perspecives. This
led to an entire family of algorithms, like \emph{Quadratic Sieve},
\emph{Dixon}, \ldots.

The core idea is still to find a pair of perfect squares whose difference can
factorize $N$, but maybe Fermat's hypotesis can be made weaker.

\paragraph{Kraitchick} was the first one popularizing the idea the instead of
looking for integers $\angular{x, y}$ such that $x^2 -y^2 = N$ it is sufficient
to look for \emph{multiples} of $N$:
\begin{align}
  x^2 - y^2 \equiv 0 \pmod{N}
\end{align}
and, once found, claim that $\gcd(N, x \pm y)$ are non-trial divisors of $N$
just as we did in \ref{sec:fermat:implementation}.
Kraitchick did not stop here: instead of trying $x^2 \equiv y^2 \pmod{N}$ he
kept the value of previous attempt, and tries to find \emph{a product} of such
values which is also a square. So we have a sequence
\begin{align}
  \label{eq:dixon:x_sequence}
  \angular{x_0, \ldots, x_k} \mid \forall i \leq k \quad x_i^2 - N
  \; \text{ is a perfect square}
\end{align}
and hence
\begin{align*}
  \prod_i (x_i^2 - N) = y^2
\end{align*}
that $\mod{N}$ is equivalent to:
\begin{align}
  \label{eq:dixon:fermat_revisited}
  y^2 \equiv \prod_i (x_i^2 - N) \equiv \big( \prod_i x_i \big) ^2 \pmod{N}
\end{align}
and voil\`a our congruence of squares (\cite{discretelogs} \S 4). For what
concerns the generation of $x_i$ with the property \ref{eq:dixon:x_sequence},
they can simply taken at random and tested using trial division.

\paragraph{Brillhart and Morrison} later proposed (\cite{morrison-brillhart}
p.187) a better approach than trial division to find such $x$. Their idea aims
to ease the enormous effort required by the trial division. In order to achieve
this. they introduce a \emph{factor base} $\factorBase$ and generate random $x$
such that $x^2 - N$ is $\factorBase$-smooth. Recalling what we anticipated in
~\ref{chap:preq}, $\factorBase$ is a precomputed set of primes
$p_i \in \naturalPrime$.
This way the complexity of generating a new $x$ is dominated by
\bigO{|\factorBase|}. Now that the right side of \ref{eq:dixon:fermat_revisited}
has been satisfied, we have to select a subset of those $x$ so that their
product can be seen as a square. Consider an \emph{exponent vector}
$v_i = (\alpha_0, \alpha_1, \ldots, \alpha_r)$ associated with each $x_i$, where
\begin{align}
  \label{eq:dixon:alphas}
  \alpha_j = \begin{cases}
    1 \quad \text{if $p_j$ divides $x_i$ to an odd power} \\
    0 \quad \text{otherwise}
  \end{cases}
\end{align}
for each $1 \leq j \leq r $. There is no need to restrict ourselves for positive
values of $x^2 -N$, so we are going to use $\alpha_0$ to indicate the sign. This
benefit has a neglegible cost: we have to add the non-prime $-1$ to our factor
base $\factorBase$.

Let now $\mathcal{M}$ be the rectangular matrix having per each $i$-th row the
$v_i$ associated to $x_i$: this way each element $m_{ij}$ will be $v_i$'s
$\alpha_j$. We are interested in finding set(s) of the subsequences of $x_i$
whose product always have even powers (\ref{eq:dixon:fermat_revisited}).
Turns out that this is equivalent to look for the set of vectors
$\{ w \mid wM = 0 \} = \ker(\mathcal{M})$ by definition of matrix multiplication
in $\mathbb{F}_2$.


\paragraph{Dixon} Morrison and Brillhart's ideas of \cite{morrison-brillhart}
were actually used for a slightly different factorization method, employing
continued fractions instead of the square difference polynomial. Dixon simply
ported these to the square problem, achieving a probabilistic factorization
method working at a computational cost asymptotically  best than all other ones
previously described: \bigO{\beta(\log N \log \log N)^{\rfrac{1}{2}}} for some
constant $\beta > 0$ \cite{dixon}.

\section{Reduction Procedure}

The following reduction procedure, extracted from ~\cite{morrison-brillhart}, is
a forward part of the Gauss-Jordan elimination algorithm (carried out from right
to left), and can be used to determine whether the set of exponent vectors is
linearly dependent.

For each $v_i$ described as above, associate a \emph{companion history vector}
$h_i = (\beta_0, \beta_1, \ldots, \beta_f)$, where for $0 \leq m \leq f$:
\begin{align*}
  \beta_m = \begin{cases}
    1 \quad \text{ if $m = i$} \\
    0 \quad \text{ otherwise}
    \end{cases}
\end{align*}
At this point, we have all data structures needed:
\\
\\
\\


\begin{center}
  \emph{Reduction Procedure}
\end{center}
\begin{enumerate}[(i)]
  \item Set $j=r$;
  \item find the ``pivot vector'', i.e. the first vector
    $e_i, \quad 0 \leq i \leq f$ such that $\alpha_j = 1$. If none is found, go
    to (iv);
  \item
    \begin{enumerate}[(a)]
      \item replace every following vector $e_m, \quad i < m \leq f$
        whose rightmost $1$ is the $j$-th component, by the sum $e_i \xor e_m$;
      \item whenever $e_m$ is replaced by $e_i \xor e_m$, replace also the
        associated history vector $h_m$ with $h_i \xor h_m$;
    \end{enumerate}
  \item Reduce $j$ by $1$. If $j \geq 0$, return to (ii); otherwise stop.
\end{enumerate}

Algorithm \ref{alg:dixon:kernel} formalizes concepts so far discussed, by
presenting a function \texttt{ker}, discovering linear dependencies in any
rectangular matrix $\mathcal{M} \in (\mathbb{F}_2)^{(f \times r)}$
and storing dependencies into a \emph{history matrix} $\mathcal{H}$.

\begin{remark}
  We are proceeding from right to left in order to conform with
  \cite{morrison-brillhart}.
  Instead, their choice lays on optimization reasons, which does
  not apply any more to a modern calculator.
\end{remark}

\begin{algorithm}
  \caption{Reduction Procedure  \label{alg:dixon:kernel}}
  \begin{algorithmic}[1]
    \Function{Ker}{$\mathcal{M}$}
    \State $\mathcal{H} \gets \texttt{Id}(f \times f)$
    \Comment the initial $\mathcal{H}$ is the identity matrix

    \For{$j = r \strong{ downto } 0$}
    \Comment reduce
      \For{$i=0 \strong{ to } f$}
        \If{$\mathcal{M}_{i, j} = 1$}
          \For{$i' = i \strong{ to } f$}
            \If{$\mathcal{M}_{i', k} = 1$}
              \State $\mathcal{M}_{i'} = \mathcal{M}_i \xor \mathcal{M}_{i'}$
              \State $\mathcal{H}_{i'} = \mathcal{H}_i \xor \mathcal{H}_{i'}$
            \EndIf
          \EndFor
          \State \strong{break}
        \EndIf
      \EndFor
    \EndFor

    \For{$i = 0 \strong{ to } f$}
    \Comment yield linear dependencies
      \If{$\mathcal{M}_i = (0, \ldots, 0)$}
        \strong{yield} $\{\mu  \mid \mathcal{H}_{i,\mu} = 1\}$
      \EndIf
    \EndFor
    \EndFunction
  \end{algorithmic}
\end{algorithm}


\section{Implementation}

Before gluing all toghether, we need one last building brick necessary for
Dixon's factorization algorithm: a \texttt{smooth}($x$) function. In our
specific case, we need a function that, given as input a number $x$, returns the
empty set $\emptyset$ if $x^2 -N$ is not $\factorBase$-smooth. Otherwise,
returns a vector $v = (\alpha_0, \ldots, \alpha_r)$ such that each $\alpha_j$ is
defined just as in \ref{eq:dixon:alphas}. Once we have established $\factorBase$, its
implementation is fairly straightforward:

\begin{algorithm}
  \caption{Discovering Smoothness}
  \begin{algorithmic}[1]
    \Require $\factorBase$, the factor base
    \Procedure{smooth}{$x$}
      \State $v \gets (\alpha_0 = 0, \ldots, \alpha_{|\factorBase|} = 0)$

      \If{$x < 0$} $\alpha_0 \gets 1$ \EndIf
      \For{$i = 1 \strong{ to } |\factorBase|$}
        \If{$\factorBase_i \nmid x$} \strong{continue} \EndIf
        \State $x \gets x// \factorBase_i$
        \State $\alpha_i \gets \alpha_i \xor 1$
      \EndFor
      \If{$x = 1$}
        \State \Return $v$
      \Else
        \State \Return \strong{nil}
      \EndIf
    \EndProcedure
  \end{algorithmic}
\end{algorithm}
\paragraph{How do we choose $\factorBase$?}
It's not easy to answer: if we choose $\factorBase$ small, we will rarely find
$x^2 -N$ \emph{smooth}. If we chose it large, attempting to factorize $x^2 -N$
with $\factorBase$ will pay the price of iterating through a large set.
\cite{Crandall} \S 6.1 finds a solution for this employng complex analytic
number theory. As a  result, the ideal value for $|\factorBase|$ is
$e^{\sqrt{\ln N \ln \ln N}}$.

\begin{algorithm}
  \caption{Dixon}
  \begin{algorithmic}[1]
    \Require $\factorBase$, the factor base
    \Function{dixon}{ }
    \State $i \gets 0$
    \State $r \gets |\factorBase| + 5$
    \Comment finding linearity requires redundance
    \While{$i < r$}
    \Comment search for suitable pairs
    \State $x_i \getsRandom \{0, \ldots N\}$
    \State $y_i \gets x_i^2 - N$
    \State $v_i \gets \texttt{smooth}(y_i)$
    \If{$v_i$} $i \gets i+1$ \EndIf
  \EndWhile
  \State $\mathcal{M} \gets \texttt{matrix}(v_0, \ldots, v_f)$
  \For{$\lambda = \{\mu_0, \ldots, \mu_k\}
    \strong{ in } \texttt{ker}(\mathcal{M})$}
  \Comment get relations
    \State $x \gets \prod\limits_{\mu \in \lambda} x_\mu \pmod{N}$
    \State $y, r \gets \dsqrt{\prod\limits_{\mu \in \lambda} y_\mu \pmod{N}}$
    \If{$\gcd(x+y, N) > 1$}
      \State $p \gets \gcd(x+y, N)$
      \State $q \gets \gcd(x-y, N)$
      \State \Return $p, q$
    \EndIf
  \EndFor
  \EndFunction
  \end{algorithmic}
\end{algorithm}

%%% Local Variables:
%%% mode: latex
%%% TeX-master: "question_authority"
%%% End:


\chapter{An Empirical Study \label{chap:empirical_study}}

Excluding Dixon's factorization method, all attacks analyzed so far exploit
some peculiarities of a candidate RSA public key $\angular{N, e}$ in order to
recover the private exponent.
Summarizingly:
\begin{itemize}
  \item Pollard's $p-1$ attack works only if the predecessor of any of
    the two primes factorizing the public modulus is composed of very small
    prime powers;
  \item  Williams' $p+1$ attack works under similar conditions - on the
    predecessor or the successor of any of the two primes ;
  \item Fermat's factorization is valuable whenever the two primes $p$ and $q$
    are really close to each other;
  \item Pollard's $\rho$ method is best whenever one of the two primes is
    strictly lower than the other;
  \item Wiener's attack is guaranteed to work on small private exponents.
\end{itemize}
Dixon's factorization method instead, being a general-purpose factorization
algorithm, can be employed to \emph{measure} the strength of a RSA
keypair: the more relations (satisfying \ref{eq:dixon:fermat_revisited}) are
found, the less it is assumed to be resistant.

Given these hypothesis, it has been fairly easy to produce valid RSA candidate
keys that can be broken using the above attacks. They have been used to assert
the correctness of the implementation.

On the top of that, there has been a chance to test the software under real
conditions: we downloaded the SSL keys (if any) of the top one million visited
websites, and survey them with the just developed software. This not only gave
us the opportunity to survey the degree of security on which the internet is
grounded today, but also led to a deeper understanding of the capacities and limits of
the most widespread libraries offering crypto nowadays.

\vfill
\section{To skim off the dataset}

What has been most scandalous above all was to discover that more than
\strong{half} of the most visited websites do \strong{not} provide SSL
connection over port 443 - reserved for HTTPS according to IANA
\cite{iana:ports}.
To put it in numbers, we are talking about $533, 000$ websites either
unresolved or unreachable in $10$ seconds.
As a side note for this, many websites (like \texttt{baidu.com} or
\texttt{qq.com}) keep a TCP connection open without writing anything to the
channel, requiring us to adopt a combination of non-blocking socket with the
\texttt{select()} system call in order to drop any empty communication.
It would be interesting to investigate more on these facts, asking ourselves how
many of those unsuccessful connections are actually wanted from the server, and
how many dropped for censorship reasons; there is enough room for another
project.

Of the remaining $450,000$ keys, $21$ were using different ciphers than RSA. All
others represent the dataset upon which we worked on.

\section{To count}

Once all valuable certificate informations have been stored inside a database,
almost any query can be performed to get a statistically valuable measure of
degree of magnitude to which some conditions are satisfied. What follows now is
a list of commented examples that we believe are relevant parameters for
understanding of how badly internet is configured today.


\begin{figure}[H]
  \includegraphics[width=0.7\textwidth]{e_count.png}
\end{figure}

The most prolific number we see here, $65537$ in hexadecimal, is the fourth
Fermat number and no other than the largest known prime of the form $2^{2^n} +
1$. Due to its composition, it has been advised by NIST as default public
exponent, and successfully implemented in most software, such as \openssl\!.

Sadly, a negligible number of websites is using low public exponents,
which makes the RSA key vulnerable to Coppersmith's attack; though, this
topic goes beyond the scope of this research and hence has not been analyzed
further.

\begin{figure}[H]
  \includegraphics[width=0.7\textwidth]{n_count.png}
\end{figure}

What is interesting to see here is that an enormous portion of our dataset
shared the same public key, pushing down the number of expected keys of one
order of magnitude. Reasons for this are mostly practical: it is extremely
frequent to have blogs hosted on third-party services such as ``Blogspot'' or
``Wordpress'' which always provide the same X.509 certificate, as they belong to
an unique organization.
Though improbable, it is even possible that exists a millesimal portion of
different websites sharing the same public key due to a
bad cryptographically secure random number generator, and therefore also the
same private key. Such a case has been already investigated in \cite{ron:whit}.

\begin{figure}[H]
  \includegraphics[width=0.6\textwidth]{localhost_certs.png}
\end{figure}

Here we go. A suprisingly consistent number of websites provides certificates
filled with dummy, wrong, or even testing informations.\\
Some do have non-printable bytes in the \emph{common name} field.\\
Some are certified from authorities. \\
Some are even gonvernmental entities.

\begin{figure}[H]
  \includegraphics[width=0.9\textwidth]{bits_count.png}
\end{figure}

According to \cite{nist:keylen_transitions} \S 3, table $2$, all RSA keys of
bitlength less than $1024$ are to be considered deprecated at the end of $2013$
and shall no more be issued since the beginning of this year. Not differently
from the above results, the remark has been globally adopted, yet still with a
few exceptions: around a dozen of non-self-signed certificates with a 1024 RSA
key appears to have been issued in 2014.


\section{The proof and the concept}

At the time of this writing, we have collected the output of only two
mathematical tests performed in the university cluster.

\paragraph{Wiener.} The attack described in chapter \ref{chap:wiener} was the
first employed, being the fastest one above all others. Recalling the different
public exponents we probed and discussed in the preceeding section (all $\leq
65537$), we expected all private expenents to be $>  \rfrac{1}{3}\sqrt[4]{N}$
and therefore not vulnerable to this particular version of Wiener's attack.
Indeed, we found no weak keys with respect to this attack. Though, as
pointed out in \cite{20years} \S 3, there is still the possibilty that the
public keys we collected could be broken employing some variants of it.

\paragraph{GCD.} On the wave of \cite{ron:whit}, whe attempted also to perform
the $\gcd$ of every possible pair of dinstinct public modulus present in the
dataset. In contrast to our expectations, this test led to no prime factor
leaked, for any key pair. We have reasons to believe this depends on the
relatively small size of our dataset, with respect to the one used in
\cite{ron:whit}.



\chapter{Conclusions \label{conclusions}}

Everytime we surf the web, we share our communication channel with lots of
entities around the globe. End-to-end encryption protocols such as TLS can
provide the security  properties that we often take as granted, like
\emph{confidentiality}, \emph{integrity}, and \emph{authenticity}; though,
these holds only if we \emph{trust} the authorities certifying the end entity.

%% Wax Taylor - Que Sera
There is this mindless thinking that whenever we see that small lock icon in the
browser's url bar, somebody is telling us the connection is safe.
There is some authority out there telling what to do, and we should be thinking
more about what these authorities are and what they are doing.
This issue is no more a technical problem, but instead is becoming more and more
a social and political problem.
It is our responsability as citzens to do something about that.


%%% Local Variables:
%%% mode: latex
%%% TeX-master: "question_authority"
%%% End:


\backmatter
%%\bibliographystyle{ieeetr}
\bibliography{library}
\clearpage
\addcontentsline{toc}{chapter}{Bibliography}

\end{document}
