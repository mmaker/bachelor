%--------------------------------------------------------------------
% Emmanuela Orsini
%--------------------------------------------------------------------

\documentclass[12pt,a4paper,twoside]{Phd_1}

% ---(Babel)--------------------------------------------------------------
%-----------------------------------------------------------------------
\usepackage[utf8]{inputenc}
\usepackage[T1]{fontenc}
\usepackage{amsmath}
\usepackage{amsfonts}
\usepackage{amssymb}
% ---(Colors)-------------------------------------------------------------
\usepackage[dvips]{color}
% or alternatively [usenames,dvips] if using ordinary LaTeX rather than pdfLaTeX
% ---(Images)-------------------------------------------------------------
\usepackage{graphicx}
\usepackage{makeidx,shortvrb,latexsym}
\usepackage{fancyhdr}
%\usepackage[boxed]{algorithm}
%\usepackage{algorithmic}
\usepackage{amsthm}
\usepackage{indentfirst}
%\usepackage{xypic}
\input xy
\xyoption{all}
%--(Tables)-----------------------------------------------------------------
\usepackage{supertabular}
%---------------------------------------------------------------------------
%%%%%%%%%%%%%%%%%%%%%%%%%%%%%%%%%%%%%%%%%%%%%%%%%%%%%%%%%%%%%%%%%%%%
%\newcommand{\pe}{\psi}
\def\d{\delta}
\def\ds{\displaystyle}
\def\e{{\epsilon}}
\def\eb{\bar{\eta}}
\def\enorm#1{\|#1\|_2}
\def\then{\;\Longrightarrow\;}
\def\Kc{{\cal K}}
\def\norm#1{\|#1\|}
\def\wb{{\bar w}}
\def\zb{{\bar z}}
\def\Bbb#1{{\mathbb #1}}
\newcommand{\GCD}{\mathrm{GCD}}
\newcommand{\lowrk}{\mathrm{low\_rk}}
\newcommand{\bestrk}{\mathrm {best\_rk}}
\newcommand{\DFT}{\mathrm {DFT}}
\newcommand{\BCH}{\mathrm {BCH}}
\newcommand{\Schaub}{\mathrm {Schaub}}
\newcommand{\SchaubPlus}{\mathrm {SchaubPlus}}
\newcommand{\s}{\cal{S}}
\newcommand{\h}{\hspace{0.4cm}}
\def\Exists{\mbox{\ exists }}
\def\And{\mbox{\ and  }}
\newcommand{\cnk}{{\mathcal C}(n,k,q)}
\newcommand{\supp}{\rm supp}
\newcommand{\fattorial}[2]{ \left( #1 \atop #2 \right) }
\newcommand{\hi}{\hspace{0.2cm}}
\newcommand{\sms}{\setminus}
\newcommand{\Special}{s2ec}
\newcommand{\spl}{s2ec}
%%%%%%%%%%%%%%%%%%%%%%%%%%%%%%%%%%%%%%%%%%%%%%%%ROBA MIA%%%%%%%%%%%
\usepackage{braket}
\newenvironment{sistema}
  {\left\lbrace\begin{array}{@{}l@{}}}
  {\end{array}\right.}
\newcommand{\dimo}{\noindent\textit{Dimostrazione.}\\}
\newtheorem{teo}[thm]{Teorema}
\newcommand{\pe}{\mathbb{P}_{\mathrm{E}}}
\newcommand{\pme}{\mathbb{P}_{\mathrm{ME}}}
\newcommand{\primary}{\blacktriangle}
\newcommand{\secondary}{{\blacktriangle\!}^{\! \! +}}
\newcommand{\MM}{\mathcal{M}}
%\newtheorem{teo}[thm]{Teorema}
\newtheorem{proposizione}[thm]{Proposizione}
\newtheorem{corollario}[thm]{Corollario}
\newtheorem{fatto}[thm]{Fatto}
\newtheorem{definizione}[thm]{Definizione}
%\newtheorem{lemma}[thm]{Lemma}
\newtheorem{osservazione}[thm]{Osservazione}
\newtheorem{esempio}[thm]{Esempio}
\newcommand{\DP}{{\D\!}^{\! \! +}}
\newcommand{\D}{\Delta}
\newcommand{\Fq}{ \mathbb{F}_{q}}
\newcommand{\Fp}{ \mathbb{F}_{p}}
\newcommand{\Lq}{ \mathcal{L}_{q}}
\newcommand{\R}{ \mathcal{R}}
\newcommand{\Cq}{ \mathcal{C}_{q}}
\newcommand{\SCa}{ S_{C,\alpha}}
\newcommand{\Cqn}{ \mathcal{C}_{q,n}}
\newcommand{\K}{ \mathbb{K}}
\newcommand{\U}{ \mathcal{U}}
\newcommand{\vi}{\mathrm{\textbf{v}}}
\newcommand{\ww}{\mathrm{w}}
\newcommand{\uu}{\mathrm{\textbf{u}}}
\newcommand{\f}{\mathsf{f}}
\newcommand{\V}{V_{\left(n,S\right)}^{\zeta}}
\newcommand{\T}{T_{\left(n,S\right)}^{\zeta}}
\newcommand{\SQ}{\begin{flushright}
$\square$
\end{flushright}}
%%%%%%%%%%%%%%%%%%%%%%%%%%%%%%%%%%%%%%%%%%%%%%%%%%%%%%%%%%%%%%%%%%%
\makeindex
\let\origdoublepage\cleardoublepage
\newcommand{\clearemptydoublepage}{%
\clearpage
{\pagestyle{empty}\origdoublepage}%
}
\let\cleardoublepage\clearemptydoublepage
\usepackage{hyperref}
% Note that colour boxes around links are NOT printed.
% The text itself can be coloured, replacing the bounding box, but
% when printing this may appear illegible;
% link colors can be set to black for printing purposes, like so:
%\hypersetup{colorlinks,%
%	citecolor=black,%
%	filecolor=black,%
%	linkcolor=black,%
%	urlcolor=black}
% Also note that this will conflict with the color package called
% earlier in this document if that is not set to the right option (e.g.
% dvips or pdflatex).
\begin{document}
%%%%%%%%%%%%%%%%%%%%%%%%%%%%%%%%%%%%%%%%%%%%%%%%%%%%%%%%%%%%%%%%
%%da riempire
%\phd  %se e' la tesi di dottorato, altrimenti non mettere nulla
\university{Universit\`{a} degli Studi di Trento}
\faculty{Facolt\`{a} di Scienze Matematiche Fisiche e Naturali}
\Logo{unitnlogo.pdf}    %metti il logo della tua universita'
\phdtitle{}     %titolo del dottorato
\title{Probabilit\`a d'errore in decodifica con un nuovo bound}  %titolo della tesi
\author{Matteo Piva}
%\twosupervisors   %scommentare se si ha anche il correlatore
\supervisor{Prof. Massimiliano Sala}
%
\firstreader{Mariello Prapapappo}  %correlatore
%
\secondreader{Prof. Sandro Mattarei}    %capo della scuola di dottorato o controrelatore
\accademico{ Anno accademico $2008/2009$}
%%%%%%%%%%%%%%%%%%%%%%%%%%%%%%%%%%%%%%%%%%%%%%%%%%%%%%%%%%%%%%%%%%%%%%%%%%%%%%%%%%%%%%
%%%%%%%%%%%%%%%%%%frontespizio%%%%%%%%%%%%%%%%%%%%%%%%%%%%%%%%%%%
\frontespizio     %questo e' il frontespizio esterno, cioe' senza firme
%%%%%%%%%%%%%%%%%%%%%%%%%%%%%%%%%%%%%%%%%%%%%%%%%%%%%%%%%%%%%%%%%%%%%%%%%%%%%%%%%%%
\cleardoublepage
%%%%%%%%%%%%%%%%%%%%%%%%%%%%%%%%%%%%%%%%%%%%%%%%%%%%%%%%%%%%%%%%%%%%%%%%%%%%%%%%%%%%
\signaturepage     %questo e' il frontesizio interno con le firme
%%%%%%%%%%%%%%%%%%%%%%%%%%%%%%%%%%%%%%%%%%%%%%%%%%%%%%%%%%%%%%%%%%%%%%%%%%%%%%%
\cleardoublepage
%%%%%%%%%%%%%%%%%%%%%%%%%%%%%%%%%%%%%%%%%%%%%%%%%%%%%%%%%%%%%%%%%%%%%%%%%%%%%%%%
%non e' ovviamente obbligatoria
%\dedica{To my grandfather C.} \dedicapage
%%%%%%%%%%%%%%%%%%%%%%%%%%%%%%%%%%%%%%%%%%%%%%%%%%%%%%%%%%%%%%%%%%%%%%%%%%%%%%%%
\cleardoublepage \setcounter{page}{1} \pagenumbering{roman}
\pagestyle{plain} \tableofcontents
%\listoffigures
%\addcontentsline{toc}{chapter}{Elenco delle figure}
%\renewcommand{\listalgorithmname}{Elenco degli algoritmi}
%\listofalgorithms
%\addcontentsline{toc}{chapter}{Elenco degli algoritmi}
%\addcontentsline{toc}{chapter}{Introduction}
%%%%%%%%%%%%%%%%%%%%%%%%%Introduzione%%%%%%%%%%%%%%%%%%%%%%%%%%%%%%%%%%%%%%%%%%%%%%%%
%\include{ack}
%\addcontentsline{toc}{chapter}{Acknowledgment}
%\cleardoublepage
%\include{Introduction}
%\addcontentsline{toc}{chapter}{Introduction}
\cleardoublepage
%---------------------------------------------------------------------
%%%%%%%%%%TESI parte centrale%%%%%%%%%%%%%%%%%%%%%%%%%%%%%%%%%%%%%%%%
\pagestyle{fancy} \pagenumbering{arabic} \mainmatter
%\part{}
%\include{chap1}
%\include{chap2}
%\part{}
%\include{chap3}
%\include{chap5}
%\part{}
%\include{chap5}
%\include{chap6}
%\part{}
%\include{chap8}
%\include{chap7}
%
%\part{Appendice }
%\include{appendice}
\backmatter
\bibliography{RefsCGC}
\addcontentsline{toc}{chapter}{Bibliografia}
\end{document}
