\chapter{Preface}

Even if the basic RSA keypair generation algorithm is fairly
straightforward, it turns out that any software willing to provide such a
feature does have to test the pair candidate against a substantious number of
tests before claiming its security.

The purpose of this project is to examine the TLS protocol, study in deep the
\openssl library, and survey some of the attacks to which a bad key generation
is exposed. On the footprint of ~\cite{20years}, we which already analyzed most
of these, we are going to describe the mathematical basis of each attack, and
then proceed further reasoning about a clever, possibly optimal, solution in
procedural programming; finally, we are trying to think about a distributed
version of it.

Besides the pseudocode already available in this document, the project led to the
development of a real, open, C implementation consultable at
\small{\url{https://github.com/mmaker/bachelor}}.

%%% Local Variables:
%%% mode: latex
%%% TeX-master: "question_authority"
%%% End:
